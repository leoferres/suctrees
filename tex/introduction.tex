Trees are ubiquitous in Computer Science.  They have applications in
every aspect of computing from XML/HTML processing to abstract syntax
trees (AST) in compilers, phylogenetic trees in computational genomics
or shortest path trees in path planning.  The ever increasing amounts
of structured, hierarchical data processed in many applications have
turned the processing of the corresponding large tree structures into
a bottleneck, particularly when they do not fit in memory.  Succinct
tree representations store trees using as few bits as possible and
thereby significantly increase the size of trees that fit in memory
while still supporting important primitive operations in constant
time.  There exist such representations that use only $2n - \Theta(\lg
n)$ bits\footnote{We use $\lg x$ to mean $\log_{2}x$ throughout this
paper.} to store the topology of a tree with $n$ nodes, which is close
to the information-theoretic lower bound and much less than the space
used by traditional pointer-based representations.

Alas, the construction of succinct trees is quite slow compared to the
construction of pointer-based representations.  Multicore parallelism
offers one possible tool to speed up the construction of succinct
trees, but little work has been done in this direction so far.  The
only results we are aware of focus on the construction of Wavelet
trees, which are used in representations of text indexes.  In
\cite{Fuentes2014}, two practical multicore algorithms for Wavelet
tree construction were introduced.  Both algorithms perform $O(n\lg
\sigma)$ work and have span $O(n)$, where $n$ is the input size and
$\sigma$ is the alphabet size.  In \cite{DBLP:journals/corr/Shun14},
Shun introduced 3 new algorithms to construct Wavelet trees in
parallel.  The in practice best algorithm performs $O(n\lg \sigma)$
work and has span $O(\lg n\lg \sigma)$.  Shun also explained how to
parallelize the construction of rank/select structures so it requires
$O(n)$ work and $O(1)$ span for rank structures, and $O(n)$ work and
$O(\lg n)$ span for select structures.

In this paper, we provide a parallel algorithm that constructs the
{\tt RMMT} tree representation of
~\cite{Navarro:2014:FFS:2620785.2601073} in $O(n/p + \lg p)$ time
using $p$ cores.  This structure uses $2n + o(n)$ bits to store an
ordinal tree with $n$ nodes and supports a rich set of basic
operations on these trees in $O(\lg n)$ time.  While this query time
is theoretically suboptimal, the {\tt RMMT} structure is simple enough
to be practical and has been verified experimentally to be very small
and support fast queries in practice~\cite{ACNSalenex10}.  Combined
with the fast parallel construction algorithm presented in this paper,
it provides an excellent tool for manipulating very large trees in
many applications.

We implemented and tested our algorithm on a number of real-world input trees
having billions of nodes.
Our experiments show that our algorithm run on a single core is competitive with
state-of-the-art sequential constructions of the {\tt RMMT} structure and
achieves good speed-up on up to 64 cores and likely beyond.

\ignore{The remainder of this paper is organized as follows:
Section~\ref{sec:relwork} gives a brief overview of the {\tt RMMT}
structure, to clearly define its structure and illustrate how it can
be used to support basic operations on trees efficiently.  It also
briefly discusses other previous work on succinct tree representations
and reviews the dynamic multithreading model, which we use to analyze
the theoretical running time of our algorithm.
Section~\ref{sec:multicoreST} describes our parallel algorithm for
constructing the {\tt RMMT} structure.  Section~\ref{sec:exps}
discusses our experimental setup and results.
Section~\ref{sec:conclusion} offers concluding remarks and discusses
future work.}

%%% Local Variables:
%%% mode: plain-tex
%%% TeX-master: t
%%% End:
