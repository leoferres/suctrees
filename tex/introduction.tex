Trees are ubiquitous in Computer Science.
We can find them in every aspect of everyday computing from XML/HTML processing
to abstract syntax trees (AST) in compilers, phylogenetic trees in computational
genomics or shortest path trees in path planning.
The ever increasing amounts of structured, hierarchical data processed in
many applications have turned the processing of the corresponding large tree
structures into a bottleneck.
The query time of these structures becomes a particular bottleneck once they
do not fit in memory any more.
To address this bottleneck, by significantly increasing the size of trees
that \emph{do} fit in memory, succinct tree representations aim to store trees
in as few bits as possible while still supporting important primitive operations
in constant time.
More precisely, there exist such representations that use only
$2n - \Theta(\lg n)$ bits\footnote{We use $\lg x$ to mean
  $\log_{2}x$ throughout this paper.} to store the topology of a tree with $n$
nodes, which is close to the information-theoretic lower bound.
This allows very large trees to be stored in memory and queried efficiently,
something that is not possible using traditional pointer-based representations.

Succinct trees, and succinct data structures in general, have a weak point,
however: their construction time.
The construction of succinct data structures is generally quite slow compared to
the construction of the corresponding pointer-based data structures.
One strategy to improve the construction time of succinct data structures
on modern computers is to exploit multicore parallelism.
Not much work has been done in this direction so far.
The only results we are aware of focus on the construction of Wavelet trees,
which are used in representations of text indexes.
In \cite{Fuentes2014}, two practical multicore algorithms
for Wavelet tree construction were introduced.
Both algorithms perform $O(n\lg \sigma)$ work and have span $O(n)$, where $n$
is the input size and $\sigma$ is the alphabet size.
In \cite{DBLP:journals/corr/Shun14}, Shun introduced 3 new algorithms to
construct Wavelet trees in parallel.
The in practice best algorithm performs $O(n\lg \sigma)$ work and has span
$O(\lg n\lg \sigma)$.
Shun also explained how to parallelize the construction of
rank/select structures so they require $O(n)$ work and span $O(1)$ for rank
structures, and $O(n)$ work and span $O(\lg n)$ for select structures.

The contribution of this paper is to provide an efficient parallel algorithm
for constructing a succinct representation of arbitrary ordinal trees that
supports a rich set of basic operations on these trees.
In particular, we discuss how to construct the {\tt RMMT} structure of
Navarro and Sadakane~\cite{Navarro:2014:FFS:2620785.2601073} in
$O(n/p + \lg p)$ time using $p$ processors.
While operations on this structure take $O(\lg n)$ time, it is simple enough
to be practical and has been verified experimentally to be very small and
support fast queries in practice~\cite{ACNSalenex10}.
Thus, combined with the fast parallel construction algorithm presented in this
paper, it provides an excellent tool for manipulating very large trees in
many applications.
Using the {\tt RMMT} structure as only a component of a more complicated
structure, Navarro and Sadakane did achieve constant-time operations but,
similar to most other succinct tree representations with constant-time
operations, it is less space-efficient than the {\tt RMMT} structure by itself
and complicated to implement, too complicated to be practical.

We implemented our algorithm and tested its performance on a number of
real-world input trees having billions of nodes.
Our experiments showed that our algorithm achieves good speed-up on up to
32 cores and likely beyond, assuming the inputs are large enough.
In earlier experiments not discussed in detail in this paper, we tested our
algorithm on smaller inputs with a few million nodes.
In these experiments, we observed poor speed-up simply because for these
trees, even a sequential construction takes less than a second, that is, there
is simply too little work to parallelize.

The remainder of this paper is organized as follows:
Section~\ref{sec:relwork} gives a brief overview of the {\tt RMMT} structure,
to clearly define its structure and illustrate how it can be used how to support
basic operations on trees efficiently.
It also briefly discusses other previous work on succinct tree representations
and reviews the dynamic multithreading model, which we use to analyze the
theoretical running time of our algorithm.
Section~\ref{sec:multicoreST} describes our parallel algorithm for constructing
the {\tt RMMT} structure.
Section~\ref{sec:exps} discusses our experimental results, including a
description of the hardware and data sets we used for the evaluation.
Section~\ref{sec:conclusion} offers concluding remarks and discusses some
future work.
