We implemented the {\tt PSTA} algorithm in C and compiled it using GCC 4.9 with
optimization level -O2 and using the \mbox{-ffast-math flag}.\footnote{The code
  and data needed to replicate our results are available at
  \url{http://www.inf.udec.cl/~josefuentes/spaa2015}.}
All parallel code was compiled using the GCC Cilk branch.
The same flags were used to compile {\tt libcds} and {\tt sdsl}, which were
written in C++.

The data sets were suffix trees of the DNA ({\tt dna}, 1,154,482,174 parentheses), and
protein ({\tt prot}, 670,721,006 parentheses) data from the Pizza \& Chili
corpus\footnote{\url{http://pizzachili.dcc.uchile.cl}}. To obtain the suffix tree,
we used the implementation provided in \url{http://www.daimi.au.dk/~mailund/suffix_tree.html}.
Besides, we used the XML trees of the Wikipedia dump\footnote{\url{http://dumps.wikimedia.org/enwiki/20150112/enwiki-20150112-pages-articles.xml.bz2} (January 12, 2015)} ({\tt wiki}, 498,753,914 parentheses) and OpenStreetMap dump\footnote{\url{http://wiki.openstreetmap.org/wiki/Planet.osm} (January 10, 2015)} ({\tt osm}, 4,675,776,358 parentheses). We also include a complete binary tree of depth 30 ({\tt ctree}, 2,147,483,644 parentheses).

The experiments were carried out on a machine with four AMD
Opteron\texttrademark{} 6278 processors clocked at 2.4GHz,
with 16 cores per processor, 64KB of L1 cache per core, 2MB of L2 cache shared by two cores,
and 6MB of L3 cache shared by 8 cores.
In total, we had 64 cores at our disposal.
The machine had 189GB of DDR3 RAM, clocked at 1333MHz.

The running times of the algorithms were measured using
the high-resolution (nanosecond) C functions in {\tt <time.h>}.
Memory consumption was measured using the tools provided by
{\tt malloc\_count} \cite{malloc-count}. In our experiments, the
chunk size $s$ and the arity $k$ were fixed at 256 and 2, respectively.
