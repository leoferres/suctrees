The {\tt PSTA} algorithm was implemented in the C programming language
and compiled using GCC 4.9 using the -O2 and -ffast-math flags. The
GCC Cilk branch (not the Cilk merged into the main GCC branch) was
used to compile the parallel code. The same flags were used to compile
{\tt libcds} and {\tt sdsl}, both written in C++. The experiments were
carried out on a quad-core AMD Opteron\texttrademark{} Processor 6278
with eight cores per processor, for a total of 32 physical cores
running at 2.40GHz. The computer runs Linux 3.11.0-26-generic in
64-bit mode. This machine has a per-core L1 and L2 caches of sizes
64KB and 2048KB, respectively and a per-processor shared L3 cache of
6MB, with a 189GB DDR RAM memory. AMD processors have a NUMA
architecture based on HyperTransport, whereby certain parts of the
code may be ``closer'' in memory than others. However, we assume the
worst case (code is farthest away) for all experiments\footnote{We
would like to thank Roberto As\'in at UCSC in Concepci\'on, Chile, for
letting us use his machine.}. Algorithms were compared in terms of
running times using the usual high-resolution (nanosecond) C functions
in {\tt <time.h>}.\footnote{This is a reproducible-research-friendly
paper, everything needed to replicate these results is available at
\url{https://github.com/jfuentess/spaa2015}}