We implemented the {\tt PSTA} algorithm in C and compiled it using GCC 4.9 with
optimization level -O2 and using the \mbox{-ffast-math flag}.\footnote{The code
  and data needed to replicate our results are available at
  \url{http://www.inf.udec.cl/~josefuentes/spaa2015}.}
All parallel code was compiled using the GCC Cilk branch.
The same flags were used to compile {\tt libcds} and {\tt sdsl}, which were
written in C++.

We tested our algorithm on five inputs.
The first two were suffix trees of the DNA ({\tt dna}, 1,154,482,174
parentheses), and protein ({\tt prot}, 670,721,006 parentheses) data from the
Pizza \& Chili corpus\footnote{\url{http://pizzachili.dcc.uchile.cl}}.
These suffix tree were constructed using code from
\url{http://www.daimi.au.dk/~mailund/suffix_tree.html}.
The next two inputs were XML trees of the Wikipedia
dump\footnote{\url{http://dumps.wikimedia.org/enwiki/20150112/enwiki-20150112-pages-articles.xml.bz2} (January 12, 2015)}
({\tt wiki}, 498,753,914 parentheses) and OpenStreetMap
dump\footnote{\url{http://wiki.openstreetmap.org/wiki/Planet.osm} (January 10,
  2015)} ({\tt osm}, 4,675,776,358 parentheses).
The final input was a complete binary tree of depth 30 ({\tt ctree},
2,147,483,644 parentheses).

The experiments were carried out on a machine with four 16-core AMD
Opteron\texttrademark{} 6278 processors clocked at 2.4GHz,
with 64KB of L1 cache per core, 2MB of L2 cache shared
between two cores, and 6MB of L3 cache shared between 8 cores.
The machine had 189GB of DDR3 RAM, clocked at 1333MHz.

The running times of the algorithms were measured using
the high-resolution (nanosecond) C functions in {\tt <time.h>}.
Memory consumption was measured using the tools provided by
{\tt malloc\_count} \cite{malloc-count}.
In our experiments, the chunk size $s$ and the arity $k$ were fixed at 256 and
2, respectively.
