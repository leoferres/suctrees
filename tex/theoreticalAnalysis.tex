The sequential version of {\tt PSTA} takes $O(N+\sqrt{2^{w}}poly(w))$,
the same complexity reported in
\cite{Navarro:2014:FFS:2620785.2601073}. The amount of work of {\tt
PSTA}, $T_1$, is the aggregation of the amount of work of Algorithms \ref{algo:PSTA1},
\ref{algo:PSTA2} and \ref{algo:PSTA2}. The work of Algorithm \ref{algo:PSTA1} is $O(N)$,
since that it behaves as a sequential list ranking algorithm (all the computation is
done in lines 8 to 26). The work of Algorithm \ref{algo:PSTA2} is $O(\frac{N}{s})$. Here,
all the computation is done in the first part (lines 1 to 18), having only one
subtree which contains all internal nodes of the {\tt RMMT}. The work of Algorithm \ref{algo:PSTA3} is equivalent to compute all universal tables sequentially, mantaining
the complexity in \cite{Navarro:2014:FFS:2620785.2601073}, that is, $O(\sqrt{2^{w}}poly(w))$. Therefore, the amount of work of {\tt PSTA} is $T_{1}=O(N+\sqrt{2^{w}}poly(w))$. Considering $p$ processors, the first part of Algorithm \ref{algo:PSTA1} (lines 5 to 26) as a complexity of $O(\frac{N}{p})$. Meanwhile, the second part (line 27) has a complexity of $O(\lg p)$ if we use the parallel list ranking algorithm in \cite{Reif1993}. Finally, the third part of Algorithm \ref{algo:PSTA1} (lines 28 to 33) has a complexity of $O(\frac{N}{sp})$. On the other hand, the first half of Algorithm \ref{algo:PSTA2} (lines 1 to 18) has a complexity of $O(\frac{Nk}{sp})$ and the second half (lines 19 to 34) has a complexity of $O(k\lg_{k}p)$. Algorithm \ref{algo:PSTA3} has a complexity of $O(\frac{\sqrt{2^{w}}poly(w)}{p})$. Therefore, the complexity of {\tt PSTA} with $p$ available processors is $T_p =
O(\frac{N}{p}+\lg p+\frac{\sqrt{2^{w}}poly(w)}{p})$, considering $s$ and $k$ constants. Now, if we consider that we have enough amount of processors, the span of {\tt PSTA} is $T_{\infty}=O(\lg N)$. Here, the span of the whole algorithm is the maximum span of all its parts. The span of Algorithm \ref{algo:PSTA1} is $O(\lg N)$, since all the computation is done in the parallel list ranking algorithm (line 27). In Algorithm \ref{algo:PSTA2} all the computation is centered in the second half, with span $O(k\lg_{k}N)$. In Algorithm \ref{algo:PSTA3}, since each cell of the universal tables can be computed independently of the rest, the span if $O(1)$. Note that the overhead implicit in each {\bf parfor} does not affect the previous complexities.

Since we have the three metrics provided by the DYM model, we can
compute speedup and parallelism. The speedup of our algorithm is $T_1/T_P$ =
$O(\frac{p(N+\sqrt{2^{w}}poly(w))}{N+\sqrt{2^{w}}poly(w)+p\lg p})$. As we can see,
under the assumption of $p\ll N$, the speedup tends to $O(p)$. The assumption
$p\ll N$ is realistic, especially considering current SMP-like
systems, where the amount of available processing units is much less than the
input, $N$ in this case. The parallelism of {\tt PSTA} is $T_1/T_{\infty}$ =
$\frac{N+\sqrt{2^{w}}poly(w)}{\lg N}$ \Jose{...}

An important advantage of our algorithm is its working space. The {\tt
PSTA} algorithm does not need any extra memory related to the usage of
threads. Indeed, it just needs a working space proportional to the
input size and the space needed to schedule the threads. A
work-stealing scheduler, as that used by the DYM model, exhibit at
most a linear expansion space, that is, $O(S_1p)$, where $S_1$ is the
minimum amount of space used by the scheduler for any execution of a
multithreaded computation using one processor. This upper bound is
optimal to within a constant factor
\cite{Blumofe:1999:SMC:324133.324234}. In summary, the working space
needed by our algorithm is $O(N+S_1p)$. Thus, considering the
assumption $p\ll N$, the working space is dominated by the input
size. The scheduling space is negligible.

The algorithm reaches its optimal behavior when the work of building
each subtree in parallel (lines 2 to 18 of Algorithm \ref{algo:PSTA2})
is the same as the work needed to create the top part of the {\tt
RMMT} (lines 19 to 32 of Algorithm \ref{algo:PSTA2}). This point is
reached when the $p\log_{k}p=N/s$. 