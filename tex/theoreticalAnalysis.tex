The sequential version of {\tt PSTA} takes $O(N+\sqrt{2^{w}}poly(w))$, the same complexity reported in \cite{Navarro:2014:FFS:2620785.2601073}. The amount of work of {\tt PSTA}, $T_1$, is also $O(N+\sqrt{2^{w}}poly(w))$, dominated mainly by the part I of our algorithm (\ref{algo:PSTA1}). The complexity of {\tt PSTA} considering $p$ processors, is $T_p = O(\frac{N}{p}+p+\frac{\sqrt{2^{w}}poly(w)}{p})$. Now, the complexity with enough amount of processors, $T_\infty$, is $O(N)$. The speedup of our algorithm is $T_1/T_P$ = $O(\frac{N+\sqrt{2^{w}}poly(w)}{\frac{N+\sqrt{2^{w}}poly(w)}{p}+p})$. As we can see, when $p$ is small, the speedup tends to $O(p)$, which meets our assumption of $p\ll N$. However, when $p$ tends to $N$, the speedup tends to decrease. This last observation is more evident in the parallelism, $T_1/T_{\infty}$ = $\frac{N+\sqrt{2^{w}}poly(w)}{N}$, which tends to $1$. The assumption $p\ll N$ is realistic, especially considering current SMP-like systems, where the amount of available processing units, even though they have increased in the last years, is much less than the input, $N$ in this case. Moreover, the size of trees have increased, having today trees with millions of nodes, such as suffix trees of text collections and XML collections.

An important advantage of our algorithm is its working space. {\tt PSTA} does not need extra memory related to the usage of threads. Indeed, it just needs a working space proportional to the input size and the space needed to schedule the threads. A work-stealing scheduler, as the used by the DYM model, exhibit at most a linear expansion space, that is, $O(S_1p)$, where $S_1$ is the minimum amount of space used by the scheduler for any execution of a multithreaded computation using one processor. This upper bound is optimal to within a constant factor \cite{Blumofe:1999:SMC:324133.324234}. In summary, the working space needed by our algorithm is $O(N+S_1p)$. Again, considering the assumption $p\ll N$, the working space is dominated by the input size and the scheduling space is negligible.

We proposed as a breakeven, a point where we can reach an optimal behavior, to our algorithm when the amount of work of each subtree that is process in parallel (lines 2 to 18 of Algorithm \ref{algo:PSTA2}) is equal to the work to create the top of the {\tt RMMT} (lines 19 to 32 of Algorithm \ref{algo:PSTA2}). This point is reached when the $p\log_{k}p=N/s$. Another possible breakeven occurrs in the Algorithm \ref{algo:PSTA1}, when the computation of partial excess values (lines 4 to 26) involves the same amount of work that to update the sequentially the final excess value of each thread (lines 27 and 28). This point is reached when $p=\sqrt{N}$. Now considering the complete algorithm, breakeven occurrs when $p\log_{k}p=N/s$, since it is reached first.