The sequential version of {\tt PSTA} takes $O(n+\sqrt{2^{w}}poly(w))$,
the same complexity reported in
\cite{Navarro:2014:FFS:2620785.2601073}. The amount of work of {\tt
PSTA}, $T_1$, is the aggregation of the amount of work of Algorithms
\ref{algo:PSTA1}, \ref{algo:PSTA2} and \ref{algo:PSTA3}. The work of
Algorithm \ref{algo:PSTA1} is $O(n)$, since that it behaves as a
sequential prefix sum algorithm (all the computation is done in
lines 8 to 26). The work of Algorithm \ref{algo:PSTA2} is
$O(n/s)$. Here, all the computation is done in the first part
(lines 1 to 18), having only one subtree which contains all internal
nodes of the {\tt RMMT}. The work of Algorithm \ref{algo:PSTA3} is
equivalent to compute all universal tables sequentially, maintaining
the complexity in \cite{Navarro:2014:FFS:2620785.2601073}, that is,
$O(\sqrt{2^{w}}poly(w))$. Therefore, the amount of work of {\tt PSTA}
is $T_{1}=O(n+\sqrt{2^{w}}poly(w))$. Considering $p$ processors, the
first part of Algorithm \ref{algo:PSTA1} (lines 5 to 26) as a
complexity of $O(n/p)$. Meanwhile, the second part (line 27)
has a complexity of $O(\lg p)$ if we use the parallel prefix sum
algorithm in \cite{Reif1993}. Finally, the third part of Algorithm
\ref{algo:PSTA1} (lines 28 to 33) has a complexity of
$O(n/sp)$. On the other hand, the first half of Algorithm
\ref{algo:PSTA2} (lines 1 to 18) has a complexity of
$O(nk/sp)$ and the second half (lines 19 to 34) has a
complexity of $O(k\lg_{k}p)$. Algorithm \ref{algo:PSTA3} has a
complexity of $O(\sqrt{2^{w}}poly(w)/p)$. Therefore, the
complexity of {\tt PSTA} with $p$ available processors is $T_p =
O(n/p+\lg p+\sqrt{2^{w}}poly(w)/p)$. Now, if we consider that we have enough amount of
processors, the span of {\tt PSTA} is $T_{\infty}=O(\lg n)$. Here, the
span of the whole algorithm is the maximum span of all its parts. The
span of Algorithm \ref{algo:PSTA1} is $O(\lg n)$, since all the
computation is done in the parallel list ranking algorithm (line
27). In Algorithm \ref{algo:PSTA2} all the computation is centered in
the second half, with span $O(k\lg_{k}n)$. In Algorithm
\ref{algo:PSTA3}, since each cell of the universal tables can be
computed independently of the rest, the span if $O(1)$. Note that the
overhead implicit in each {\bf parfor} does not affect the previous
complexities.

Given the DYM model, the speedup of our algorithm is $T_1/T_p$ =
$O(\frac{p(n+\sqrt{2^{w}}poly(w))}{n+\sqrt{2^{w}}poly(w)+p\lg
p})$. Under the assumption of $p\ll n$, the speedup tends to
$O(p)$. In turn, the
parallelism $T_1/T_{\infty}$ (the maximum theoretical speedup) of {\tt
PSTA} is $\frac{N+\sqrt{2^{w}}poly(w)}{\lg N}$.

An important advantage of our algorithm is its working space. The {\tt
PSTA} algorithm does not need any extra memory related to the use of
threads. Indeed, it just needs space proportional to the input size
and the space needed to schedule the threads. A work-stealing
scheduler, like the one used by the DYM model, exhibits at most a
linear expansion space, that is, $O(S_1p)$, where $S_1$ is the minimum
amount of space used by the scheduler for any execution of a
multithreaded computation using one processor. This upper bound is
optimal to within a constant factor
\cite{Blumofe:1999:SMC:324133.324234}. In summary, the working space
needed by our algorithm is $O(n\lg n+S_1p)$. Thus, since in modern
machines it is usual that $p\ll n$, the scheduling space is negligible
and the working space is dominated by the input size.

% \Jose{The algorithm reaches its optimal behavior when the work of
% building each subtree in parallel (lines 2 to 18 of Algorithm
% \ref{algo:PSTA2}) is the same as the work needed to create the top
% part of the {\tt RMMT} (lines 19 to 32 of Algorithm
% \ref{algo:PSTA2}). This point is reached when the
% $p\log_{k}p=N/s$. REVIEW THIS PART}