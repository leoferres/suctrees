{\em Dynamic multithreading} (DYM) \cite[Chapter 27]{Cormen2009} is a
model of parallel computation which is faithful to several industry standards
such as Intel's CilkPlus (\url{cilkplus.org}) and OpenMP Tasks
(\url{openmp.org/wp}), and Threading Building
Blocks (\url{threadingbuildingblocks.org}).

We will define a {\em multithreaded computation} as a directed acyclic
graph (DAG) $G=(V,E)$, where the set of vertices $V$ are instructions
and $(u,v) \in E$ are dependencies between instructions; whereby in
this case, $u$ must be executed before $v$.\footnote{Notice that the
  RAM model is a subset of the DYM model where the outdegree of every
  vertex $v \in V$ is $\leq 1$.} In order to signal parallel
execution, we will augment sequential pseudocode with three keywords,
{\bf spawn}, {\bf sync} and {\bf parfor}. The {\bf spawn} keyword
signals that the procedure call that it precedes {\em may be} executed
in parallel with the next instruction in the instance that executes
the {\bf spawn}. In turn, the {\bf sync} keyword signals that all
spawned procedures must finish before proceeding with the next
instruction in the stream. Finally, {\bf parfor} is simply ``syntactic
sugar'' for {\bf spawn}'ing and {\bf sync}'ing ranges of a loop
iteration, adding an overhead which is logarithmic in the number of
iterations. If a stream of instructions does not contain one of the
above keywords, or a {\bf return} (which implicitly {\bf sync}'s) from
a procedure, we will group these instruction into a single {\em
  strand}. Strands are scheduled onto processors using a {\em
  work-stealing} scheduler, which does the load-balancing of the
computations. Work-stealing schedulers have been proved to be a factor
of 2 away from optimal performance
\cite{Blumofe:1999:SMC:324133.324234}.

To measure the efficiency of our parallel algorithm, we
will use three metrics: the {\em work}, the {\em span} and the number
of processors. In accordance to the parallel literature, we will
subscript running times by $P$, so $T_P$ is the running time of an
algorithm on $P$ processors. The {\em work} is the total running time
taken by all (unit-time) strands when executing on a {\em single}
processor (i.e., $T_1$),\footnote{Notice, again, that analyzing the
  work amounts to finding the running time of the serial algorithm
  using the RAM model.} while the {\em span}, denoted as $T_\infty$,
is the {\em critical path} (the longest path) of $G$. In this work,
we are interested in speeding up the succinct tree construction and finding
out the upper bounds of this speedup. To measure this, we will define
{\em speedup} as $T_1/T_P = O(P)$, where linear speedup $T_1/T_P =
\Theta(P)$, is the goal. We also define {\em parallelism} as the ratio
$T_1/T_{\infty}$, the maximum theoretical speedup that can be achieved
on {\em any} number of processors.