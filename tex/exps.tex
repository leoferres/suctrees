To evaluate the performance of our {\tt PSTA} algorithm, we compare it against
{\tt libcds}~\cite{libcds} and {\tt sdsl}~\cite{sdsl}, which are
state-of-the-art implementations of the {\tt RMMT}.
Both assume that the input tree is given as a parenthesis sequence, as we do
here.
Our implementation of the {\tt PSTA} algorithm deviates from the description in
Section~\ref{subsec:multicoreSTAlgorithm} in that the prefix sum computation in
line~27 of the algorithm is done sequentially in our implementation.
This changes the running time of the algorithm to $O(n/p + p)$ but simplifies
the implementation.
Since $p \ll n/p$ for the input sizes we are interested in and the numbers of
processors available on current multicore systems, this simplification has an
insignificant impact on the running time of our algorithm.
%Moreover, the results of Patrascu explained in section \ref{subsec:idea} were
%not implemented, because they are not practical.\Norbert{Patrascu is not mentioned
%anywhere in Section 3.1 nor in the references. ???}
%Available implementations of the {\tt RMMT} do not implement this part either.

%NOTE: Here I was talking about aB-trees. I decided to delete this part, because anyway
%we (and statr-of-the-art implementations) did not implement them.