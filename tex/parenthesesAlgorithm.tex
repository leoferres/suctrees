As we mentioned above, the input to {\tt RMMT} consists of a
parentheses sequence $P$. However, sometimes $P$ is not available,
instead we have un ordinal tree, $T$. The construction of $P$ from $T$
has not been considered in previous implementations of the {\tt
  RMMT}. Here, we proposed an algorithm to compute $P$ in parallel,
complementing the {\tt PSTA} algorithm. We will call this algorithm
\emph{Parallel Folklore Enconding Algorithm}, {\tt PFEA}. The key
observation to obtain $P$ is that each parenthesis associated to an
edge $(u,v)$, ``('' or ``)'', is followed in the enconding by a
parenthesis associated to the edge $(parent(u), v)$, $(v, first(v))$
or $(u, next(u,v))$. All of them at distance of 1 of the edge
$(u,v)$. In the observation, $parent(u)$ is the parent node $u$,
$first(v)$ is the first child of $v$ and $next(u,v)$ is the child of
$u$ that succeeds the node $v$ in the array or children. With this
observation in mind, the core of {\tt PFEA} works as follows: Let
$(u,v)$ be an edge of $T$, where $u$ is parent of $v$ and let
$(u,v)_{0}$ ($(u,v)_{1}$) the opened (closed) parenthesis associated
to $(u,v)$. First, if $v$ is a leaf, then $(u,v)_{0}$ is followed by
$(u,v)_{1}$. If $v$ is not a leaf, then $(u,v)_{0}$ is followed by
$(v,first(v))_{0}$. If $v$ is not the last child of $u$,then
$(u,v)_{1}$ is followed by $(u,next(u,v))_{0}$. Finally, if $v$ is the
last child of $u$, then $(u,v)_{1}$ is followed by
$(u,parent(u))_{1}$. 

\begin{algorithm}[t]
\small
\SetVlineSkip{-2cm}
  % keywords
  \SetKwInOut{Input}{Input}
  \SetKwInOut{output}{output}
  \SetKwFor{PFor}{parfor}{do}{end}
  \LinesNumbered
  \SetAlgoNoEnd
  \DontPrintSemicolon
  % I/o
  \Input{Arrays $V$ and $E$ representing the ordinal tree $T$}
  \output{Parentheses sequence representation of $T$}
  \BlankLine% \SetAlgoNoLine
  % algorithm
  $ET, P$ = arrays of size $2|E|$\;
  \PFor{$i\leftarrow 0$ \KwTo $|E|-1$}{
  	$orig = E[i].origin$,
  	$dest = E[i].destiny$\;
  	$ET[2i].value = 0$\tcp*[h]{(}\;
  	$ET[2i+1].value = 1$\tcp*[h]{)}\;
  	
  	\eIf{$dest.node$ is a $leaf$}{
  		$ET[2i].succ = ET[2i+1]$\;
	}
	{
  		$ET[2i].succ = ET[2*next(dest.node,dest.index)]$\;
    }
  \BlankLine
  	\eIf{$orig$ is a last incident egde}{
  		$ET[2i+1].succ = ET[2*parent(orig.node)+1]$\;
	}
	{
  		$ET[2i+1].succ = ET[2*next(orig.node,orig.index)]$\;
    }
  }
  \BlankLine
  $parallel\_list\_ranking(ET)$\;
  \BlankLine
  \PFor{$i\leftarrow 0$ \KwTo $2|E|-1$}{
  	$P[ET[i].rank] = ET[i].value$
  }
  \caption{{\tt PFEA}}
  \label{algo:PFEA}
\end{algorithm}
\normalsize

